\chapter*{ABSTRACT}

\addcontentsline{toc}{chapter}{Abstract}

This dissertation presents a feasibility study on the application of Trigger-Object Level Analysis (TLA) to the search for the Standard Model Higgs boson produced by Vector Boson Fusion (VBF) and decaying to b-quarks, using $4.6$fb$^{-1}$ of proton-proton collision data taken at a centre-of-mass energy of $13$TeV by the ATLAS detector. The VBF process is predicted to be the second largest cross-section process at the Large Hadron Collider, and searches for the VBF produced Higgs decaying to $b\bar{b}$ exploit the characteristic final state topology to select events. TLA refers to the procedure of only storing the jet objects reconstructed at the trigger-level of the ATLAS detector for use in analysis, which permits an increase in the output rate of the detector, as a result of reducing the byte size of the detector readout.

This dissertation suggests that a TLA approach is feasible for the \VBFHBB\ channel. The analysis showed that the behaviour of trigger-level jets was comparable to that of the standard reconstructed jets, and the agreement of the jet kinematic properties could be improved by applying trigger-level calibrations. Simulating the \VBFHBB\ analysis using trigger-level objects resulted in a $20\%$ reduction in the number of final events for the TLA compared to the standard analysis, which shows that the $100\%$ detector output rate increase from the current TLA approach would produce an overall increase in the number of final state events. Select kinematic properties of the \VBFHBB\ final state were studied and shown to demonstrate comparable behaviour between the trigger-level and standard reconstructed objects.

\cleardoublepage

\endinput
