\chapter{Conclusions}
\label{c:c}

This dissertation contains work done to test the feasibility of using Trigger-Object Level Analysis to improve the statistical significance of searches for the Higgs boson produced via Vector Boson Fusion and decaying to bottom quarks. This feasibility was tested by assessing the performance of trigger-level online analysis compared to the reconstructed offline analysis at two levels of abstraction for the \VBFHBB\ event, firstly at the resolution of comparing the performance for the individual jet objects that make up a \VBFHBB\ event separate from the \VBFHBB\ topology, and then by performing elements of the full \VBFHBB\ analysis with both online and offline objects to compare the performance.

The analysis was carried out using a vector boson fusion Monte-Carlo simulation sample and $4.63$fb$^{-1}$ of data collected by the ATLAS detector during data-taking period D of the 2016 $\sqrt{s}=13$TeV Run.

The individual online and offline jet objects were shown to be comparable to each other, and could be improved to show a closer agreement in behaviour using standard ATLAS analysis tools. The positions of the \bjets\ and non-\bjets\ that make up a \VBFHBB\ event were shown to agree within $1\%$ of each other in ($\eta$, $\phi$) space. The \pt distributions of the online and offline jets demonstrated small differences, with offline \pt being $\sim5\%$ larger than online \pt for both the \bjets\ and the non-\bjets. This \pt difference arises from a difference in the jet energy scale calibrations of the online and offline jet objects, and can be rectified in future analyses using standard jet calibration tools.

The \btag\ performance of the individual jet objects were compared, which showed differences in the \btag\ efficiency, \cjet\ rejection and light-jet rejection between the online and offline jets. These differences in efficiency were consistent with the expected change in performance  resulting from the different \btag\ algorithms applied to the online and offline jets. The 2016 MV2c10 \btag\ algorithm to the offline reconstructed jets, while the \btag\ information for the online jets was calculated using the 2015 MV2c20 algorithm that was operational in the detector at that time. This suggests \btag\ performance of the online objects can be brought into agreement if the \btag\ training variables are preserved on the trigger-level objects, rather than being discarded and leaving only the \btag\ decision in the DxAOD.

Comparison of the online and offline performance in a \VBFHBB\ event phase space was then carried out. These cuts resulted in a final online event count that was reduced relative to the offline event count, with a final online event fraction of $82\%$ for Monte-Carlo simulation and $84\%$ for data. With the increased trigger rate permitted by using TLA, this would on average increase the final event number by $66\%$ relative to a purely offline analysis.

Finally the \VBFHBB\ event specific objects, kinematic quantities and BDT training variables were compared for the online and offline events. For each separate variable, the performance of the online analysis was broadly consistent with respect to the offline analysis, taking into account the reduction in the number of online events highlighted during the cutflow analysis. These results suggest that TLA analysis in the \VBFHBB\ channel will provide increased statistical significance while providing comparable events to the full offline reconstructed analysis.

The work of this dissertation suggests certain additional studies outside the scope of this analysis should be carried out prior to approving TLA for the \VBFHBB\ channel. Primarily, the practicalities of applying TLA in the \VBFHBB\ channel require assessment. The solutions proposed in this dissertation to improve the agreement between the online and offline objects will increase the size of the trigger-objects output by the detector and may result in additional computational cost in the HLT. This may result in a smaller rate increase than assumed based on prior TLA studies and reduce or remove the improvement in rate suggested here.

In addition, the comparative behaviour of the online and offline objects could have further verification steps. This work did not implement trigger emulation in the Monte-Carlo simulations, and this resulted in discrepancies between the Monte-Carlo and data results for the jet object performance. The full \VBFHBB\ analysis performed at $\sqrt{s}=8$TeV carried out a BDT analysis after the cuts implemented in this dissertation to enhance the \VBFHBB\ phase space. Implementing or retraining a BDT was not possible within this dissertation, but would be an informative branch of further work to verify the feasibility. Finally, technical limitations prohibited making use of the full data set produced by the ATLAS detector, so analysis was carried out on a subset of the data. Greater statistical significance and a more certain statement of similarity could be made by performing the analysis for a larger dataset.

This study on the feasibilty of performing TLA on the \VBFHBB\ channel search for the Higgs boson suggests that the trigger-level objects used for a \VBFHBB\ analysis are comparable to the offline objects, and that the similarity can be improved with some readily available calibrations and adjustments to the trigger-level objects. Also, the final \VBFHBB\ event produced using trigger-level objects will show a worse efficiency compared to offline reconstruction, but with the trigger rate increase afforded by TLA produce more events than an offline analysis. These additional events will be comparable in behaviour to the offline reconstruction. There are some additional sections of work relating to implementing and completely verifying the conclusions of this dissertation, but overall trigger-object level analysis is suggested as a feasable analysis strategy in the search for the Higgs boson via the \VBFHBB\ channel.
