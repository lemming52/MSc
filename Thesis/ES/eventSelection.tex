\chapter{Event Selection}\label{c:ES}

NOTES:
what do we need in this sectionL:
	xAODs
	offline recovery 
	online recover
	central channels
	triggers used
	
	
	\section{Samples}
		 Real event data was taken from the 2016 13 TeV run, with Data Period D used owing to limited storage space on analysis computing facilities. The analysis used the all year 25ns Good Runs List, \verb|data16_13TeV.periodAllYear_DetStatus-v88-pro20-21_DQDefects-00-02-04_PHYS_StandardGRL_All_Good_25ns.xml|, resulting a a data luminosity of ?????????? fb$^{-1}$. The simulated VBF sample used, \verb|mc15_13TeV.341566.PowhegPythia8EvtGen_CT10_AZNLOCTEQ6L1_VBFH125_bb.merge.DAOD_HIGG5D3.e3988_s2726_r7772_r7676_p2719|, was produced during the MC15c production period. The sample uses the HIGG5D3 xAOD derivation as described in Ref. \cite{HIGG5D3}. The sample was produced using the NLO generator \textsc{powheg} configured using the CTEQ6L1 \cite{CTEQ} set of PDFs and interfaced with \textsc{pythia8} tuned to AZNLO \cite{AZNLO}.
		 
	\section{\VBFHBB Analysis Strategy} 
		As higgs bosons are colour singlets, there is no colour charge connection between the \bquarks produced in the interaction and the other final state particles, which leads to little hadronic activity and QCD radiation. This results in the two VBF jets being produced with a high rapidity gap, the pair of which along with the central \bjets form the characteristic final state for the interaction. With this in mind, the \VBFHBB events are separated from the high QCD multijet backgroud by requiring two central \bjets which form the higgs candidate and two high \pt non \btagged  VBF jets. The analysis is split into two separate channels, related two details of the topology and the trigger chains used \todo{cite 2017 vbfhbb paper}. The \textit{four-central} channel requires all four jets to be contained within the central region $|\eta| < 2.8$ and the \textit{two-central} channel requires two jets in the central region and one forward jet. In this study, the online trigger level jets could not be extracted for the \textit{four-central} trigger chains, so analysis focusses on the \textit{two-central} channel. 
		
		For the \textit{two-central} channel, the event was required to pass the \verb|HLT_j80_bmv2c2070_split_j60_bmv2c2085_split_j45_320eta490| trigger. The event was required to contain one jet with \pt$>95$GeV which was \btagged at the \textit{tight} working point and one additional jet with \pt$>70$GeV that passed the \textit{loose} \btag working point. One forward jet with $3.2 < |\eta| < 4.4$ and \pt$>60$GeV was required along with a final VBF jet with \pt$>20$GeV and $|\eta| < 4.4$. Finally the \pt of the $bb$ pair was required to exceed 160 GeV. (This is to remove sculpting EXPLAINNNN)
		
	\section{Jet Extraction}
	
		The analysis is based on the jet objects reconstructed from the detector contained in the DxAOD. Both the offline jet objects and the online equivalents are retrieved, however the method by which the full collection of jets is assembled differs between each case. For offline jet objects, the DxAOD contains a complete set of jets for each reconstruction algorithm, which are each associated to the relevant jet btagging information. 
		
		In selecting the trigger level offline jets, firstly all \textit{split-jets} that pass the trigger are retrieved from the trigger chain. Any duplicates, determined through $\Delta R$ matching, are removed and the \btag information is associated with the jets. Following this all L1 trigger jets are retrieved, which do not possess \btag information. The full set of L1 jets is compared to the \textit{split-jets} and any duplicates are removed from the HLT jet set to form the \textit{nonsplit-jets}. The combination of the \textit{split-jets} and \textit{nonsplit-jets} forms the complete jet collection for the trigger level event.
		%As the \textit{split-jets} are required to have been \btagged, they are limited to a range of $|\eta| < 2.8$\todo{confirm btagging range} and the forward jet is by necessity in the \textit{nonsplit-jets}  #
		
	\section{Jet Assignment}
	
		For both online and offline studies the jet collections are processed to extract the four \VBFHBB jets. Firstly separate collections of jets that pass the \textit{loose} \btag working point and lower \bjet \pt cut and jets that do not pass the \btag but pass the lower VBF jet cut are assembled. 
	

This section describes the selection criteria required for the events and reconstructed objects used in the analysis. These cuts and criteria are designed with the \VBFHBB event topology in mind, along with the limitations introduced by considering the available trigger chains as discussed in Section \ref{VBF:TriggerChains}. These cuts are applied in the \VBFHBB analysis and the direct object comparison covered in Chapter \ref{c:OP}.

\section{Events}

Data events were required to pass the all year 25ns Good Runs List\footnote{\detokenize{data16_13TeV.periodAllYear_DetStatus-v88-pro20-21_DQDefects-00-02-04_PHYS_StandardGRL_All_Good_25ns.xml}} \REF{GRL} and also be Clean \REF{Clean}. 

\section{Offline Jets}

	Offline jet reconstruction was performed by the anti-$k_t$ algorithm (R=0.4) as discussed in Section \ref{t:jetReco}. Jets were calibrated in line with the 20.7 recommendations \REF{jets:calib}. When considering individual jets during the analysis, all jets were required to have a \pt $> 45$ GeV to be recorded.
	
\section{Online Jets}
	
	Online Jet reconstruction is a mystery. A full collection of online jets was recovered by extracting the split jets (Section \ref{det:split) and L1 Jets (Section \ref{det:trigger:L1}). The full set of online jets was considered as the recovered split jets combined with any L1 jets that did not match any split jet. When considering individual jets during the analysis, all jets were required to have a \pt $> 45$ GeV to be recorded. \todo{This needs to be understood}
	
\section{Offline \textit{b}-jets}

	The specifics of $b$-tagging are covered in Section \ref{det:btagging}. Offline $b$-jets were tagged using the \textit{MV2c10}-tagger\footnote{Jan 2017 Recommendations: \detokenize{2016-20_7-13TeV-MC15-CDI-2017-01-31_v1.root}} with two defined efficiency working points: \textit{Tight}, with an overall efficiency of 70\% and \textit{Loose} with 85\% tagging efficiency. 
	
\section{Online \textit{b}-jets}
	
	 Online $b$-jets were tagged using the \textit{MV2c20}-tagger\footnote{Mar 2016 Recommendations: \detokenize{2016-Winter-13TeV-MC15-CDI-March10_v1.root}} with two defined efficiency working points: \textit{Tight}, with an overall efficiency of 70\% and \textit{Loose} with 85\% tagging efficiency.   
 

\endinput
