\chapter{Event Selection}\label{c:ES}

	This chapter is describes the selection criteria for real and simulated event data, along with the specific calibrations and configurations used in the extraction and reconstruction of the objects making up the analysis. The event selections described here were chosen to target the analysis towards the typical \VBFHBB\, final state described in Section \ref{t:VBF}.

	\section{ATLAS Event Data}

	The raw data from the ATLAS detector is stored in a proprietary data format used by the ATLAS experiment, the Analysis Object Data (AOD) format. This is the output of the event reconstruction software, with each event having a corresponding discrete entry. For Run-2 of the LHC experiment, this was upgraded to the xAOD format, which is readable by ROOT \cite{ROOT}, a modular software framework managed by CERN and designed specifically for analysis of large datasets with complex statistical analysis, visualisation of data and storage. The xAOD format is a many leveled branching tree structure, with nodes of the tree grouping together related information from each event, and has an associated Event Data Model (EDM) to standardise classes, interfaces and types for representation of an event facilitating simple analysis \cite{xAOD}.

	Analyses typical make use of a derivation framework to refine the complete xAOD into a more selective Derived xAOD (DxAOD) which will normally only the relevant objects to a target analysis, and results in a smaller dataset that is much easier to manipulate, store and operate over. These derivations are produced using the ATLAS bulk data processing framework Athena \cite{athena}. The computation framework used for analysis of the xAOD dat is the internally developed Analysis Base suite of tools. This analysis uses Analysis Base Release \texttt{2.4.31} and made use of the EventLoop package for event processing.

	This set of tools is used for both the real event data and the simulated Monte-Carlo data, with DxAODs of both datasets forming the core data for any ATLAS physics analysis. These datasets, following from the large output rate of the LHC, are extremely large, necessitating the use of parallelised computation to perform any statistically significant analysis. The computational framework developed at ATLAS is designed to perform concurrent computation, and processing, making use of the Worldwide LHC Computing Grid \cite{grid} to provide the necessary hardware capacity.


	\section{Event weights}

		 In order to accurately compare the simulated events from the Monte-Carlo samples with the real event dataset, it is necessary to normalise the Monte-Carlo samples to the total luminosity of the dataset, based on the theoretical cross-section for the interaction. The Monte-Carlo simulation assigns a weight $w_i$ to each event simulated, which are summed to give the total number of events in the Monte-Carlo. Each bin of any histogram in the results produced from the simulated data is reweighted using a scaling factor:

		 \begin{equation}
		 w_{MC} = \frac{\sigma k L}{N}
		 \end{equation}

		 where $\sigma$ is the theoretical cross section, $L$ the integrated luminosity of the real dataset, $N$ the total number of simulated events ($\Sigma_N w_i$) and $k$ the Real $K$-Factor, which is a correction to the leading order cross section to reproduce the higher order calculation for the interaction.


	\section{Samples}
		 Real event data was taken from the 2016 13 TeV run, with Data Period D used owing to limited storage space on analysis computing facilities. The \texttt{HIGG5D3} derivation was used for the data and Monte-Carlo samples, with a full list of tags given in Appendix \ref{a:config}. This analysis used the all year 25ns Good Runs List (Table \ref{t:files}), resulting a a data luminosity of $4.6312$fb$^{-1}$. The simulated VBF sample (Table \ref{t:files}) was produced during the MC15c production period. This sample was produced using the NLO generator \textsc{powheg} configured using the CTEQ6L1 \cite{CTEQ} set of PDFs and interfaced with \textsc{pythia8} tuned to AZNLO \cite{AZNLO}.

	\section{Jet Extraction}

		The analysis is based on the jet objects from the detector contained in the DxAOD, the reconstruction of which is covered in Section \ref{d:jetreco}. Both the offline jet objects and the online equivalents are retrieved, however the method by which the full collection of jets is assembled differs in either case. For offline jet objects, the DxAOD contains a complete set of jets for each reconstruction algorithm, which are each associated to the relevant jet \btag\, information. Offline jets were calibrated in line with the 20.7 recommendations (Table \ref{t:config}). In addition, given the high \pt cuts required for an event, as discussed in Section \ref{es:as}, all jets were required to have \pt$>45$GeV.

		In selecting the trigger level offline jets, firstly all \textit{split}-jets that pass the trigger are retrieved from the trigger chain. Any duplicates, determined through $\Delta R$ matching, are removed and the \btag\, information, which is stored in a separate xAOD container, is associated with the jets. Following this all L1 trigger jets are retrieved, which do not possess \btag\, information. The full set of L1 jets is compared to the \textit{split}-jets and any duplicates are removed from the L1 jet set to form the \textit{nonsplit}-jets. The combination of the \textit{split}-jets and \textit{nonsplit}-jets forms the complete jet collection for the trigger level event.

		When searching for \bjets\, or forward jets in the complete jet collection for online jets, only the \textit{split}-jets can be considered for \bjetsas they are associated with \btag information. Both \textit{split} and \textit{nonsplit}-jets can be considered for the VBF jet during jet assignment as describing in Section \ref{es:as}.

		\subsection{\bjets}

		The details of \bjet identification are covered in Section \ref{det:btagging}. Offline $b$-jets were tagged using the \textit{MV2c10}-tagger configured using the January 2017 recommendations (Table \ref{t:config}) with two defined efficiency working points: \textit{tight}, with an overall efficiency of 70\% and \textit{loose} with 85\% tagging efficiency. Online $b$-jets were tagged using the \textit{MV2c20}-tagger as configured during the data taking, which made use of the March 2016 Recommendations (Table \ref{t:config}) with two identically defined \textit{tight} and \textit{loose} working points.


	\section{\VBFHBB\, Analysis Strategy}
	\label{es:as}
		 Following from the description of the \VBFHBB\, events in Section \ref{t:VBF}, target events are selected by requiring two central \bjets\, which form the Higgs candidate and two high \pt VBF jets. Searches using \VBFHBB\, consider two exclusive analysis channels of interesting events: the \textit{four-central} channel, which requires all four jets to be contained within the central region $|\eta| < 2.8$, and the \textit{two-central} channel which requires two jets in the central region and one forward jet. In this study, the online trigger level jets could not be extracted for the specific trigger chains used previously for the \textit{four-central} channel, so analysis focuses on the \textit{two-central} channel.

		For the \textit{two-central} channel, the event was required to pass the \texttt{HLT\_j80\_bmv2c2070\_split\_\-j60\_bmv2c2085\_split\_j45\_320eta490} trigger. This trigger requires a single L1 jet ROI of $E_\text{T} > 40$GeV and $|\eta| < 2.5$. In addition, a second central jet ROI with $E_\text{T} > 25$ and a forward jet ROI with $E_\text{T} > 20$GeV and $3.1 < |\eta| < 4.9$ are both required.
		At the HLT level, one central jet \btagged\, at the \textit{tight} working point with \pt $>80$GeV, and a jet with \pt$>60$GeV tagged at the \textit{loose} working point were both required. Finally a HLT forward jet with $E_\text{t}>45$ between $3.2 < |\eta| < 4.9$ was needed.

		Once the trigger was passed, the event was required to contain one jet with \pt$>95$GeV which was \btagged\, at the \textit{tight} working point and one additional jet with \pt$>70$GeV that passed the \textit{loose} \btag\, working point. One forward jet with $3.2 < |\eta| < 4.4$ and $p_{\text{T}}>60$GeV was required along with a final VBF jet with \pt$>20$GeV and $|\eta| < 4.4$. Finally the \pt of the $bb$ pair was required to exceed 160 GeV. This cut is to remove kinetic sculpting of the $M_{bb}$ distribution, which for absent or lower \ptbb cuts has a pronounced bump in the $200$-$300$GeV $M_{bb}$ region. This bump is a result of the correlation between \mbb and \ptbb, with the \ptbb distribution featuring a peak as a result of the individual jet \pt requirements. By requiring the \ptbb cut the \mbb distribution forms a regular falling distribution.

		The events were required to be clean events, unaffected by any small detector issues, and the jets were assigned to components of the \VBFHBB\, event as described in the following procedure. All pairs of jets that passed the \textit{loose} working point where either of the jet pair passed the \textit{tight} working point were considered; the pair with the highest \ptbb was selected as the Higgs candidate and designated \bjets $b_1$ and $b_2$ with respect to individual \pt. An identical iterative procedure was carried out to assign the VBF pair, using jets not marked for consideration as the Higgs candidate. One of the VBF jet pair was required to satisfy the forward jet selection criterion, and the highest invariant mass pair was selected and labeled $j_1$, $j_2$ according to \pt.

		These conditions were identical for both the Monte-Carlo and real data samples, with the exception of the trigger requirements which were not required for the Monte-Carlo samples.

		In a full analysis, the signal is extracted from the results using a Boosted Decision Tree trained to extract the \VBFHBB\, events from the \ggF\, contributions. Time constraints in this analysis prohibited a full BDT analysis, but discussion of boosted decision trees and training is covered in Appendix \ref{a:bdt}.


\endinput
