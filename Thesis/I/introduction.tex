\chapter{Introduction}
\label{c:intro}

Modern understanding of particle physics is best described by the Standard Model (SM), a theoretical framework describing the behaviour and interactions of all known fundamental particles. The Standard Model covers three of the four fundamental forces, omitting gravitational interactions, and has been thoroughly explored with decades of experimental observations showing good agreement with its predictions. Within the theory of the Standard Model, it is postulated that particles acquire mass by interacting with the Higgs field. This acquisition of mass occurs via spontaneous breaking of the underlying gauge invariant symmetries that make up the Standard Model framework.

The proposal of this Higgs field in the 1960s led to the consideration of a new scalar particle, the Higgs boson, formed from excitations of the field \cite{gauge-boson-mass, higgs-1, higgs-2}. However, experimental evidence for the Higgs boson remained a significant missing component of the Standard Model for decades. Providing this evidence was one of the primary reasons for the construction of the Large Hadron Collider \cite{lhc} (LHC) at the  Conseil Europ\'{e}an pour la Recherche Nucl\'{e}aire (CERN). The LHC is a proton-proton collider designed with a centre-of-mass energy of $14$~TeV, built to test the experimental predictions of the Standard Model and look for New Physics in areas beyond the Standard Model. In 2012 the ATLAS and CMS collaborations, two separate general purpose particle detector experiments at the LHC, announced the observation of a new particle in the search for the Standard Model Higgs boson \cite{cmshiggs, atlashiggs}.

\newpage
Following on from the initial discovery, additional studies have been undertaken to probe this new particle and establish if it is consistent with the Standard Model Higgs boson. With these new studies and the increased experimental dataset provided by the continued running of the LHC; the spin, mass and couplings of the new particle have been shown to be consistent with the SM Higgs, and current measurements give the Higgs mass $m_H=125.09\pm0.29$~GeV \cite{higgsmeasure}.

There are several distinct production mechanisms proposed for a Higgs boson at the LHC. Of these mechanisms, the vector boson fusion (VBF) process is expected to have the second largest cross-section \cite{LHCHiggsCS}, and with a Higgs mass of $\sim125$~GeV, the dominant decay mode expected \cite{HDECAY} is the $H\rightarrow b\bar{b}$ mode. Measurement of the cross-section of \VBFHBB\ provides important information on the properties and behaviour of the Higgs boson, probing the strength of the VBF interaction and the coupling of the Higgs boson to down type quarks specifically. Analysis of the \VBFHBB\ has already been carried out by both the CMS \cite{cmsvbfhbb} and ATLAS \cite{VBFHbb8tev} collaborations on LHC proton-proton collision data at a centre-of-mass energy of $8$~TeV. Studying this channel is complicated by the large background contributions from multijet events among other background sources, which necessitates limiting the trigger rate and reduces the number of relevant events.

At the LHC, the interaction rate far exceeds the possible data output rate, which is limited by the available bandwidth of the machine. As a result, the detector output relies on triggers to identify and record events of interest, which operate with pre-scaling reductions on event rates to reduce their output to within the bandwidth constraints. This results in significant numbers of discarded events for topologies lacking distinctive easy to detect signatures that can be used as a trigger. To overcome this limitation, the Trigger-Object Level Analysis strategy was proposed, where rather than storing the complete detector readout, the jet reconstruction information used in the triggering system is output and used for the physics analysis. This reduces the size of the detector output and allows the event output rate to be increased while remaining within bandwidth limitations. Such a TLA has been performed successfully, with a corresponding increase in rate, for the search for light dijet resonances at ATLAS \cite{tla}.

The objective of this dissertation was to test the feasibility of applying a TLA to the search for the Higgs boson in the \VBFHBB\ channel. This was done by comparing the behaviour of trigger-level objects with standard analysis reconstructed objects, individually and with reference to the \VBFHBB\ topology. The analysis used $4.6$~fb$^{-1}$ of data taken during Run-2 of the LHC in 2016 at a centre-of-mass energy of $13$~TeV.

\newpage
An overview of the theoretical framework of the Standard Model and other physics relevant to the \VBFHBB\ interaction is given in Chapter \ref{c:Theory}. The experimental details of the ATLAS detector at the LHC are discussed in Chapter \ref{c:Det}, while the specific details of the ATLAS data and analysis procedure for \VBFHBB\ are given in Chapter \ref{c:ES}. Chapter \ref{c:OP} covers comparison of individual trigger-level objects to standard reconstructed objects and Chapter \ref{c:K} contains similar comparisons with consideration of the overall \VBFHBB\ event. Finally, the conclusions of this dissertation are presented in Chapter \ref{c:c}.


