\documentclass[]{article}

\usepackage{amsmath}
\setlength{\parindent}{0em}
\setlength{\parskip}{1em}
\usepackage[none]{hyphenat}
\usepackage{fancyhdr}
\lhead{PHYS40591 - Radio Astronomy}
\pagestyle{fancy}

\usepackage[top=1.5cm, bottom=1.5cm, left=3cm, right=3cm]{geometry}
\begin{document}
\large

\section*{Overall Notes}

Course Structure
\begin{itemize}
	\item Emission Mechanisms
\end{itemize}

Recommended Book:

\section{26/9/16}

"Lecture 1" is non-examinable.

\subsection{Emission}

Specific Intensity ($I_v$), a measure of the power emitted within a particular frequency bandwidth by a region into a solid angle of 'detector'. Power emitted $P_{em} = I_vdAd\nu d\omega$. In this were are assuming a source at infinity.

Surface Brightness ($B_v$), measure of power received by a collector from an emitter. $B_v = \frac{flux}{solid angle}$ per unit $\nu = \frac{dE}{dtdAd\omega\nu}$, $P_{rec} = B_vdAd\nu d\omega$.

These values are effectively the same, with a couple of caveats. Surface brightness is a conserved property. Caveats arise owing to the physical nature of reality. Surface brightness is a property determined at the source, while Specific intensity is determined at receiver. As such, an system that interacts with the radiation between source and detector will affect the specific intensity and prevent the equivalence. Factors are obvious, absorption, scattering, curvature of the universe, diffraction.

 

\end{document}


