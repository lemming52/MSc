\documentclass[]{article}

\usepackage{amsmath}
\setlength{\parindent}{0em}
\setlength{\parskip}{1em}
\usepackage[none]{hyphenat}
\usepackage{fancyhdr}
\lhead{PHYS40521 - Frontiers of Particle Physics 1}
\pagestyle{fancy}

\usepackage[top=1.5cm, bottom=1.5cm, left=3cm, right=3cm]{geometry}
\begin{document}
\large

\section*{Overall Notes}

CP Violation: Results in the M/AM asymmetry in the universe.

Course Structure
\begin{itemize}
	\item Introduction to modern particle physics experiments
	\item Physics at the high energy frontier.
	\item 1. Standard model and the 3 frontiers
	\item 2. Particle properties and experimental methods in particle physics
	\item 3. Modern-day experiments.
	\item 4. data analysis, stats and Monte-Carlo techniques.
\end{itemize}

Books:
\begin{itemize}
	\item Particle Physics: Martin and Shaw
	\item Particle Data Group Website: pdg.lbl.gov
\end{itemize}

\section{28/9/16: SM, Frontiers}

\subsection{Standard Model}

A minimal theoretical framework to describe the weak, EM and strong interactions. Theory of 'almost everything', but doesn't include gravitation, dark matter or dark energy. Not a finished theory, and is known to be incomplete.

Particles:
Fermions, quarks and leptons with spin 1/2.
Bosons, force mediators spin 1 and higgs with spin 0.
Grouped into generations.

Interactions mediated by the gauge bosons, $g, Z^0, W^\pm, \gamma$. Gauge theory of $SU_C(3) x SU_L(2) x U_\gamma1)$.

\begin{itemize}
	\item SU(3) is QCD, strong interactions.
	\item EM interactions mediated by photons, QED.
	\item Weak interactions mediated by Z and W, unified with QED at higher enery scale.
	\item EW Symmetry breaking which gives mass to Z and W described by Higgs Mechanism.
\end{itemize}

Underlying theory is QFT. Each fermion interacts differently. Quarks interact with all forces (s, EM, W), electrons interact with (EM and W) and neutrinos only interact via the weal interaction..

Each particle has a fundamental mass arising from coupling to the Higgs field, and all of these interactions are significantly higher magnitude than gravitation.


\subsection{Feynmann Diagrams}
Tree level diagrams have no closed loops, Loop diagrams contain virtual particle loops and any particle interaction has an infinite number of loop diagrams.

\subsection{QED}
Each vertex contributes a term to the matrix element proportional to the coupling constant $\alpha = \frac{e^2}{4\pi\epsilon_0\hbar c} \approx\frac{1}{137}$. Higher order diagrams are suppressed. New particles and interacions can contribute to loops modifying predictions.

\subsection{QCD}
Describe interaction of quarks via massless gluon exchange. Has to include gluon self interaction, which is Non-Abelian, as both quaks and the mediator gluons carry colour charge. 

The coupling constant for strong interactions runs in terms of energy. Calculations only converge if the constant $\alpha_s \ll 1$.

QCD does however have Asymptotic freedom. The coupling constant decreases with energy. The source of this decrease with energy is the self interaction of the gluons. As a result, to study QCD we need to work at high-energy.

As mentioned, gluons are not colour neutral. Massless gluons have infinite range, which would require infinite energy for self interacting long range fields. This is solved by requiring a physical particle to be colourless, and quarks are confined. 

This manifests as hadronic jets.

\subsection{Weak Interaction}
Mediated by self interacting massive $W, Z$ bosons. Managed by electroweak unification, prediction of weak neutral currents mediated by Z. Strength of EQ interactions given by weak mixing angle, $cos^2\theta_W = \frac{M_W^2}{M_Z^2}$.

\subsection{Particle/Field Formulation in SM}

Classical dynamics used the Lagrangian, in particle physics use a lagrangian while using he Dirac Equation to describe free fermions as fields (spinors)

\begin{equation}
L = \bar{\Psi}(i\gamma^\mu\partial_\mu - m)\Psi
\end{equation}

By taking the Schrodinger equation and requiring relativistic invariance, Dirac predicted antiparticles.

\subsection{Standard Model Tests}
\begin{itemize}
	\item LEP: electroweak and QCD
	\item Fermilab: Electroweak, QCD, quark mixing.
	\item LHC (running)
	\item LBNF(DUNE) (planned)
	\item International Linear Collider
\end{itemize}

Higgs Production:
\begin{itemize}
	\item Gluon Fusion
	\item Vector boson fusion
	\item $t\bar{t}$ fusion
	\item Associated production, Higgsstralum
\end{itemize}

\subsection{Constraints of SM}

Doesn't predict:
\begin{itemize}
	\item Fine Structure Constant
	\item Weinburg angle
	\item Strong coupling
	\item Higgs mass
	\item Z boson mass
	\item CKM mixing angles(3) + phase
	\item quark masses (6)
	\item charged lepton masses (3)
\end{itemize}


These are all just numbers effectively that are not predicted in the model but are placed in the model by hand. Including neutrinos add mixing angles and masses for 7 additional parameters.

The model does predict relationships, all observables can be predicted in terms of the free parameters. If sufficient measurements are made, SM can be over-constrained. Over-constrain: don't have any more 'ad hoc' inpu and can test consistency model. Practically, pick well measured set of observables, calculate other observables from that then measure.

GUT Scale:
The grand unification scale ($10^{16}GeV$) The standard model provides no explanation for interactions at gigantic energy scales.

Dark matter and energy:
Not involved even slightly. 

Fundamental Particle Masses:
The generations, etc.

Neutrino Masses

Gravity:
No way of describing GR within Quantum Field Theory. Would require a quantum with spin 2, the graviton. gravitational Waves.

\subsection{Solving The Problems}

Three major options: increase the energy of particle accelerators so new particles are indicative of a new sector
Increase intensity of accelerator to perform precision measurements for rare or forbidden decays
Non accelerator particle physics, cosmic rays, proton decay.

\section{5/10/16: }

	\subsection{Particle detection and identification}
	
		Detection relies on observing particle induced effets in some instrumented detector mmedium.
		This relies on the particle properties, energy, momentum, qunatum numers, mass.
		
		Particles transfer energy to the medium they travel through (silicon, He gas, Liquid Xe, CaW0$_4$), typically in a gradual and stochastic fashion.
		The transfer can be sudden anc complete (pair productoin, inelastic collision) which is a destructive collision.
		Non destructive detection and measurement ot $E, p$ allow the particle to be located on the path.
		
	\subsection{Charged and Neutral particles}
	
		Aim to identify as many products as possible.
		
		\begin{itemize}
			\item Stable charged particles - traverse all or part of the detector - $e^-, \mu^-, \pi, K, p$
			\item Stable neutral particles - ma be directly detected - $\gamma, n, K^0_L$ mesons.
			\item Long lived particles - decay vertices may be reconstructed - $K^0_S$ mesons, $c_\tau=2.64$ cm, $\Lambda$ baryons $c_\tau=7.89$ cm
			\item Short-lived ($c_\tau=100-500\mu$m) particles - decay vertices may be reconstructed in high-resolution silicon vertex detectors - mesons an baryons containing $b, c$ quarks $\tau$ leptons.
			\item Particles that are produced but cannot be detected directly:
				\begin{itemize}
					\item neutrinos -  reconstructed from missing energy and momentum
					\item unstable particles - missing $E, p$ - reconstruct the invariant mass of unstable parents from the decay products
				\end{itemize}
		\end{itemize}
		
		Processes for detecting charged particles:
		\begin{itemize}
			\item ionisation, excitation (scintillation), \u{C}erenkov radiation, transition radiation, bremmsstrahlung (breaking).
			\item all involve gradual $E$ loss via multiple interactions
		\end{itemize}
		
		Processes for photons:
		\begin{itemize}
			\item Photoelectric effecct, Compton scattering, pair production.
			\item in pp, a photon loses all it's energ in a singe interaction.
			\item PE and compton dominate at low $E$, pp at high.
		\end{itemize}
		
		Processes for hadrons(charged and neutral)
		\begin{itemize}
			\item Neuclear elastic scattering and inelastic interactions
			\item Main pprocesses for hadronic showers.
		\end{itemize}
		
	\subsection{Energy Loss by Ionisation}
	
	When caused by high $E$ particles in dectors can identify measure flu and properties. Does influence the propagation. 
	
	Elastic collisions liberate quasi-free atomic electrons that can be detected and measured.
	
	The Bethe-Bloch formula for the rate of loss of energy by a charged particle in a medium.
	
	\begin{equation}
	-\langle\frac{dE}{dx}\rangle = Kz^2\frac{Z}{A}\frac{1}{\beta^2}[\frac{1}{2}\ln\frac{2m_ec^2\beta^2\gamma^2Tmax}{I^2} - \beta^2 - \frac{\delta(\beta\gamma)}{2}]
	\end{equation}
	
	This assumes $M \gg m_e$ the process is complicated for low mass owing to relativistic complications. There are two term clusters, a high effect cluster outside the log, and the log section that gradually changes. Describes energy loss of relativistic particles in mater, depending strongly on $z$ incoming charge, $\beta = \frac{v}{c}$, material $Z, A$. logarithmically on $I$ ionisation energy, $\gamma$ boost factor.
	
	The main features can be derived classically.
	
	Derivation:
	
	Consider incident particle mass $M$, charge $ze$ at velocity $v$ in medium with $e^-$ density $n$, where $e^-$ are free and at rest.
	
	The momentum transfer to the electron is the integral of the electrostatic force over the time of passage of the particle. Longitudinal components cancel, and overall results is $\Delta p_\perp = \frac{2ze^2}{bv}$ omitting constant factors and where $b$ is impact parameter of the electron.
	
	A single electron gets $\Delta p$, so classical $E_k = \frac{\Delta p^2}{2m_e}$. Now consider a cylindrical barrel. 

\end{document}


