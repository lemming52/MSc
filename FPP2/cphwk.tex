\documentclass[]{article}

\usepackage{amsmath}
\setlength{\parindent}{0em}
\setlength{\parskip}{1em}
\usepackage[none]{hyphenat}
\usepackage{fancyhdr}
\lhead{A.Strange FPP2}
\pagestyle{fancy}

\usepackage[top=1.5cm, bottom=1.5cm, left=3cm, right=3cm]{geometry}
\begin{document}
\large

\section{ Experimental evidence for ratio of number of baryons to photons being $10^{-9}$, implications for particle composition for transition between radiation and matter dominated universe.}

	The ratio can be determined from nucleosynthesis. From our understanding of nucleosynthesis, it is shown the abundance of light elements in the universe is dependent on the ratio of Baryon/photon. Nucleosynthesis refers to the production of nuclei other than $^1H$ in the early stages ($O(10s)$) of the universe. The value of the baryon/photon ratio is an indication of the reaction rate of nucleosynthesis processes in the early universe, and by comparing the predictions of the rate to the observed distribution of light elements in the observable universe in the present day this value can be probed.
	
	In addition, the ratio can also be determined by considering the cosmic microwave background. By calculating the number density of photons using the blackbody distribution, and considering present day temperature of space. By estimating the baryon density using observation of space, the ratio can be calculated.
	
	The value of this ratio, as stated, tunes the abundance of light elements following the early universe through the radiation dominated epoch to the matter dominated stage. Obviously, the value implies photons far outnumbered baryons in the universe, but the value is more significant on the relative proportions of elements such as $^4$He and deuterium, if the baryon density was higher, more reactions would occur and the helium density would increase.
	
\section{Sakharov conditions, and their necessity.}

	When considering the early universe, if matter and antimatter are formed in equal proportions, to arrive at the state of the current universe, somewhere along the path between the initial and current states, there must be some degree of asymmetry that favours matter in order to produce the matter dominated universe we observe. We know the universe is dominated by matter from our general observations and also from the absence of prevent distinctive signatures (i.e. intense light from annihilation along intersecting boundaries) that would indicate the presence of equally uniform antimatter regions within the universe.
	
	Antimatter's existence is strongly predicted by quantum mechanics and relativity, along with backings of experimental observations, and no primordial anti matter is observed, so some disparity is required.
	
	To explain this baryon asymmetric universe, three requirements were proposed by Sakharov:
	
	\begin{itemize}
		\item Require some process that violates baryon number.
		\item A breaking of C and CP symmetries (charge conjugation and parity symmetries)/ C and CP violation
		\item These conditions must occur during a phase where the universe is not in thermal equilibrium
	\end{itemize}
	
\section{Why can the weak interaction be the only interaction to mediate transition from u to d type quarks}

	Simply, the force carrier of the weak interaction, the W$^\pm$ boson is the only force carrier with charge. With conservation of charge in a process, the only way a u and d type quark could interact is if the charge difference between the two was accounted for, as is done by the weak interaction.
	
	Weak current is charged, so permits flavour change.
	
\section{}
	
\section{}

\newpage

\section{CP Violation from the weak interaction}

	The feature that allows for CP violation over the weak interaction is the difference between the weak eigenstates and the mass eigenstates of quarks. Cabbibo's theory regarding the rotation matrix between the strong and weak eigenstates, when extended into the third generation using the CKM matrix permits CP violation.
	
	As stated the CKM matrix relates the weak states to the mass states, with the matrix elements related to the probabilities of state transitions (this makes the matrix unitary). While making it a matrix of probabilities the amplitude is limited such that the matrix is unitary, there is freedom to change the phases of states if one considers complex matrix elements. With the CKM matrix, there are 9 amplitudes and nine phases, one of each for each element. With the nine amplitudes constrained by the unitary requirement, and 5 of the phases capable of being quark relative phases, we are left with 4 unconstrained phases. This works out to 3 cabibbo angles and 1 spare phase, with which CP can be violated. 
	
	This explanation is a little backward to its derivation, as the third family of quarks was predicted from CP violation rather than this inverted logic.
	
\section{Non-zero electron dipole moment of Neutron}

	If the neutron possessed a non-zero dipole moment, there would be violation of both CP and Time reversal in terms of symmetry.
	
	The neutron already possesses a non zero magnetic dipole moment. Considering first time reversal, if we include a non-zero electric dipole moment, under time reversal, the magnetic dipole moment/spin of the particle will change reverse, while the electric moment remains constant. This results is a state different from the initial state, hence violates T-reversal symmetry. 
	
	For CP violation, the explanation is similar. Under parity inversion, the charge flips so electric dipole moment reverses. But, the magnetic dipole moment is not changed, and again we end up with a different state to the initial.
	
\section{J/$\psi K^0_s$}

\section{$|V_{us}|$, $|V_{cs}|$}

	In both cases, the branching ratio of a decay in question is dependent upon the matrix element we wish to study, and so by studying the branching fraction the value can be extracted.

	\subsection{$|V_{us}|$}
	
	There are two approaches to determining $|V_{us}|$. One relies on theoretical calculations and the other phenomenological measurements.
	
	The principle decays for the $|V_{us}|$ are semi-leptonic, both based around $s\rightarrow\pi e\nu$. (This approach has issues: the final state of the decay is dominated by one state, and theoretical calculations prefer a dense variety of final states, there are qcd corrections that render a perturbative expansion meaningless and running quark masses introduce uncertainty)
		
		\begin{equation*}
			K^+ \rightarrow \pi^0e^+\nu_e
		\end{equation*}
		\begin{equation*}
			K^0_L \rightarrow \pi^-e^+\nu_e
		\end{equation*}
		
	The other experimental approach relies on averaging the branching ratios of kaon decays across several experiments and include calculation of a form factor from an 'unquenched lattice QCD calculation'
	
	\subsection{$|V_{cs}|$}
	
	Measurement of $|V_{cs}|$ can be done via measurements of leptonic $D^+$ and $D^+_s$ decays
	
	\newpage
	
\section{$\tau^- decay$}
	 
\section{`The' Unitary Triangle}

	`The' unitary triangle refers to the easiest to measure, the db triangle.
	
	\begin{equation*}
	V_{ud}V^*_{ub} + V_{cd}V^*_{cb} + V_{td}V^*_{tb} = 0
	\end{equation*}

\newpage
\section{Wolfenstein Parametrisation: CKM}

		$\begin{pmatrix}
			1 - \frac{\lambda^2}{2} & \lambda & A\lambda^3(\rho - i\eta) \\
			-\lambda & 1 - \frac{\lambda^2}{2} & A\lambda^2 \\
			A\lambda^3(1 - \rho - i\eta) & -A\lambda^2 & 1 
		\end{pmatrix}$
		
		$\lambda \sim 0.22$, $A \sim 1$
		
\section{Wolfenstein Parametrisation: Unitary triangle}
	
	\begin{align*}
	V_{ud}V^*_{ub} + V_{cd}V^*_{cb} + V_{td}V^*_{tb} &= 0 \\
	(1 - \frac{\lambda^2}{2}) \times (A\lambda^3(\rho - i\eta))^* + (-\lambda)\times(A\lambda^2)^* + A\lambda^3(1 - \rho - i\eta) &= 0.	
	\end{align*}
	
	\newpage
	
\section{CP Violation in the B meson system}

	As covered, there are three variants of CP violation: 
	
	\begin{itemize}
		\item Direct CP violation / CP violation in decay
		\item CP violation in mixing
		\item CP violation in the interference between decays of mixed and unmixed mesons
	\end{itemize}
	
	Direct CP violation can be readily observed in the two body $B^0$ and $B_S$ decays, examples:
	
	\begin{align*}
	B^0_s &\rightarrow \pi K \\
	\bar{B}^0 &\rightarrow \pi^+K^-.
	\end{align*}
	
	Interference measurements include measurements of the $B \rightarrow D^0 K$ decay or the $B^0 \rightarrow \pi\pi$ mode.
	
	The measurement of the semi-leptonic B Asymmetry is a measurement of CP Violation in mixing, relying on spotting a discrepancy is the numbers of opposing like sign lepton pairs  produced  after the production of a $B\bar{B}$ pair (this is the result of oscillation of the B mesons).
	
\section{}

\section{}

\section{}


\newpage
\section{Dalitz Plot}

	Dalitz plots are used for analysis of three body decays, and study the manner in which the three bodies are produced. Typically plots are made with each axis being the invariant mass squared of one of the three pairs of products. If there is no correlation between the products, the distribution of these variables with be uniform within kinematic constraints on the system, but resonances or correlations will appear in non-uniform sections within the distribution, peaked around the invariant mass of the resonance.
	
	An example of usage is in the study of $B^\pm \rightarrow hhh$, for example $B^+\rightarrow K^+\pi^+\pi^-$. For CP violation, plots can be made for each of $B^\pm$ and then compared against one another. Any difference, which may be local rather than across the plot, represents CP violation.
	
	For this particular plot, one would expect to see resonances cause be $K^*$, $\rho^0$, $f^0$ and $\chi_{c0}$, represented as dark bands in the plot.
	
\section{$B_s\rightarrow\mu\mu$}

	The $B_s\rightarrow\mu\mu$ is a very useful decay, while it is exceedingly rare $O(10^{-9})$, with produces a distinct experimental signature in the 2 muons that is easy to spot and use as a trigger in a detector. The interaction is well described by the standard model, either as a box or a penguin diagram, and is sensitive to new physics.
	
	Its sensitivity arises as new physics models (e.g. SUSY with a charged higgs) predict a branching ratio that is strongly dependent ($O(param^6)$ for SUSY) on constraints in the new physics models. However, the LHCb experiment found a value for the result that is consistent with the SM prediction and placed a string constraint on SUSY. 
	
	 
	
	
	
	


\end{document}
