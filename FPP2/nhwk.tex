\documentclass[]{article}

\usepackage{amsmath}
\setlength{\parindent}{0em}
\setlength{\parskip}{1em}
\usepackage[none]{hyphenat}
\usepackage{fancyhdr}
\lhead{A.Strange FPP2}
\pagestyle{fancy}

\usepackage[top=1.5cm, bottom=1.5cm, left=3cm, right=3cm]{geometry}
\begin{document}
\large

\section{Neutrino Sources}

	\begin{itemize}
		\item Atmospheric: $\nu_e, \nu_\mu$, Wide energy range, typical studied range $O(0.1-100)$GeV.
		\item Solar: $\nu_e$, $E=O(1-10)$MeV.
		\item Supernovae:, $\nu_e$, $E=O(10)$MeV.
		\item High Energy Neutrino Sources (Astronomical particle accelerators): All types, $E=O(10^{21})$eV.
		\item Earth's Core: $\bar{\nu}_e$, $E=O(1-10)$MeV.
		\item Accelerator beams:  $\nu_\mu$, $E=O(0.1-100)$GeV.
		\item Reactor: $\bar{\nu}_e$, $E=O(1-10)$MeV.
		
	\end{itemize}

\section{Neutrino Oscillation Experiments}

	\begin{itemize}
		\item MINOS: Sensitive to $\nu_e$, $\nu_\mu$, Measure $\theta_{23}$, $\Delta m^2_{23}$. Process $\nu_\mu\rightarrow\nu_\tau$
		\item Super-Kamiokande: $\nu_e$, $\nu_\mu$; $\theta_{23}$, $\Delta m^2_{23}$; $\nu_\mu\rightarrow\nu_\tau$
		\item K2K: $\nu_e$, $\nu_\mu$; $\theta_{23}$, $\Delta m^2_{23}$; $\nu_\mu\rightarrow\nu_\tau$
		\item OPERA: $\nu_\tau$;  $\theta_{13}$, $\theta_{23}$, $\Delta m^2_{23}$;  $\nu_\mu\rightarrow\nu_\tau$,  $\nu_\mu\rightarrow\nu_e$
		\item T2K:  $\nu_e$, $\nu_\mu$; $\theta_{13}$, $\theta_{23}$, $\Delta m^2_{23}$;  $\nu_\mu\rightarrow\nu_\tau$,  $\nu_\mu\rightarrow\nu_e$
		\item NOVA: $\nu_e$; $\theta_{13}$;  $\nu_\mu\rightarrow\nu_e$
		\item Kamland: $\bar{\nu}_e$; $\Delta m^2_{12}$,  $\theta_{13}$ ; $\nu_\mu\rightarrow\nu_e$
	\end{itemize}
	
\section{Neutrino Mass Constraints}

	\begin{itemize}
		\item $m(\nu_\tau) < 18.2$MeV, derived from kinematics of $\tau^-\rightarrow n(\pi)\nu_\tau$
		\item $m(\nu_\mu) < 0.19$MeV, derived from kinematics of stopped $\pi^+\rightarrow \mu^+\nu_\mu$
		\item $m(\nu_\mu) < 2.3$eV, derived from the radioactive beta decay of tritium which has a very small emitted energy, used in the KATRIN experiment, explained below.
		\item absolute mass $\Sigma m_\nu < $0.23 eV, measured from cosmological studies: the PLANCK experiment, work of the Sloan Survey and studies of Supernovae.
		
		\subsection{KATRIN experiment}
			In a beta decay, when including the neutrino, the neutrino has at minimum $m_\nu c^2$ of energy. Measurement of this case, where the neutrino has minimal momentum is the eta decay end point measurement, so named as in the spectrum of electron energies it occurs at the end point of the decreasing curve. this is the approach used in the KATRIN experiment.
			
			The endpoint is determined by measuring the highest energy electrons produced by beta decay, produced electrons are sent though a retarding electric field, selecting by kinetic energy for the spectrometer. The upper limit on the electron energy observed provides the lower limit on the mass of the neutrino.
			
			Tritium is used specifically as it is the simplest nucleus that shows beta decay, making modelling of the mass change between the decay source and product as easy as possible, and also the recoil energy of the nucleus can be easily accounted for. in addition, with the low energy of the decay less slowing of the electrons is required and the breadth of the energy spectrum is smaller, so finding the endpoint is a little easier. Measurements are scheduled to begin soon.
		
	\end{itemize}
	
\section{Two Flavour Oscillation}
\newpage

\section{PMNS}
\newpage

\section{Cowan-Reines}

The Cowan-Reines experiment, while originally proposed to use a nuclear detonation, made use of a nuclear reactor to produce electron antineutrino flux. These were detected using two water tanks, via detection of gamma rays produced by annihilation of positrons produced by neutrino interactions with protons. the tanks were stacked on top of each other, with scintillators separating and surrounding the tanks to detect the gamma rays. The experiment was improved by adding cadmium to the water to interact with produced neutrons from the neutrino interaction, which would then produce a gamma ray. The experiment was configured such that the cadmium gamma was produced with a delay compared to the annihilation. 

To ensure background sources were minimised, the experiment was based in a shielded location underground. The signature of the interaction helps cut down on background interactions. In the positron annihilation, both scintillators around a water tank will flash, as a result of the pair of gamma rays from the same source. This helps reduce the effect of contaminating gamma rays, as require both scintillators to react. The delayed cadmium gamma ray adds an extra step of verification, requiring a double then single flash for an event.

Cosmic rays were excluded using the double tank stack formation. If a cosmic ray interacted, it would appear in both tanks, and all three scintillators. A neutrino interaction would only occur in one tank, so the third scintillator would not show a response, vetoing the interaction. 

\section{OPERA Detector}

	\subsection{a}

		The OPERA detector is an emulsion based detector. This consists of layers of 'photographic' emulsion placed between alternating steel (acting as a neutrino interaction target) and plastic layers. While this permits significant spatial resolution, there is no digital readout as tracks are preserved in the emulsion which must be removed, developed and analysed.
		
		In the $\nu_\tau$ interaction, the incident neutrino interacts with a nucleus to produce a $\tau^-$ lepton. This decays rapidly, owing to the very short lifetime of the $\tau^-$ to a $\nu_\tau$ and to some charged particle. In the emulsion, this results in a characteristic signature of the sudden appearance of a charged particle, which has a very short track of a few mm, before there is another interaction and a different charged track is observed. This initial kink is the distinctive feature for the $\nu_\tau$
	   interaction.
	
\subsection{b.) }

\subsection{c.) }

\newpage
		
\section{Neutrino Beam}

	\subsection{a.) }
	
		The typical neutrino beam first consists of a proton accelerator. This is aimed towards a hadron production target, which produces a stream of hadrons, principally pions. As neutrinos cannot be focused once they are produced, the hadrons produced are focused, normally using a magnetic horn, a device that selects by charge the relevant pions (and by definition, rejects the opposing charge) and forms a narrow beam. These pions will then decay to muons and muon neutrinos, and detectors can be places along the narrow beam and whatever distance is desired.
		
	\subsection{b.)}
	
		Contamination can arise from decays of other mesons produced from the target, along with occasional three body muon decay in the narrow beam and contamination from external sources. This contamination is crucial to determining the mixing angle. The expected number of observed transitions is extremely low (notable experiments achieved $o(10)$ events), so any degree of contamination will have a significant affect.
		
\section{Neutrino Interactions}
\newpage
		
\section{MINOS}

	\subsection{a.) }
	
		The MINOS detector consists of two detectors, near and far. The near detector is smaller, placed near a graphite target used to produce the muon neutrino bean using colliding protons and magnetic horns. The further detector is larger, consists of scintillator calorimeters made of alternating plates of steel and plastic scintillators within a magnetic field. The magnetic field induces curvature in products of neutrino interactions which aids identification.
		
		This curvature allows the ability to  distinguish between matter and antimatter as required in the search for CP violation.
		
		In terms of operating principle, the basic point is comparing flux rate of neutrinos between the near and far detectors in order to probe the oscillation between neutrino states occurring in transit.
		
	\subsection{b.) }
	
		These hypothetical fourth state neutrinos would have no electric charge, and so cannot interact via the weak interaction.
		
	\subsection{c.) }
	\newpage
	
	\subsection{d.) }
	
	From the derived oscillation equation, in particular the $sin^2$ term not relating to the mixing angle, it can be clearly seen that increasing the mass splitting that is in the numerator will result in requiring a larger energy to see a difference due to the oscillation and also a decrease in the frequency of oscillation with energy.

\section{Dark Matter}

	\subsection{a.)}

	Evidence for the presence of dark matter:
	
	\begin{itemize}
		\item Motion of galaxies within a galaxy cluster suggests significant increase in present mass when compared with estimates based on observable bodies. Work of Fritz Zwicky.
		\item Galaxy rotation curves. The distribution of orbital velocities of stars within galaxies, when modelled with most matter in a disc, was found to differ from the velocity curve that would be expected based on observable matter, and consistent with an expansive dark matter halo enveloping the galaxies. Work of Vera Rubin.
		\item Galactic Lensing observations. Observation of galactic lensing effects on distant bodies being traversed by high mass objects, (galactic clusters) suggesting additional mass.
		\item Velocity Dispersion: Similar to the galactic rotation curves, the distribution of velocities within the rings of elliptical galaxies does not match observational estimates, and can be reconciled by introducing dark matter.
		\item Cosmic Microwave Background. While significantly uniform, the anisotropies present in the CMB can be analysed and attributed distinctly to regions of standard or dark matter, owing to the non-interaction of dark matter with radiation.
		\item 'Standard Candles', supernovae. While a backwards approach based on the other mysterious component of the observable universe, dark energy, estimates of the expansion of the universe based on the supernovae leave a discrepancy in the energy density of the universe that can be filled by considering some material that behaves in a  a similar way to matter.
		
	\end{itemize}
	
	WIMP: Weakly Interacting Massive Particle. Interact solely through the weak nuclear force and gravity, with the possibility of interacting via other interactions with cross-section values at maximum comparable with the weak force. would be required to be larger in mass when compared to the standard particles of the standard model. No EM interaction, hence dark.
	
	Particles matching these set of criteria are the prime candidates for the 'missing matter' which can explain the observations before, as they would not be visible via astronomical methods, and the weakly interacting aspect would make detecting them extremely difficult, consistent with our total lack of any observed dark matter candidate particle.
	
	\subsection{b.) }
	
	CDMS, Cryogenic Dark Matter Search, experiment designed to attempt to directly detect any dark matter WIMP particles. The detector consists of silicon or Germanium disks cooled to millikelvins, in an attempt to measure ionization and phonons (QM vibration) from any WIMP interactions.
	
	The ratio of the ionisation and phonon signals can be used to distinguish between interactions with the atomic electrons or the atomic nuclei (WIMPS predicted to interact with the nuclei, while most background signals interact with electron). The design also allows separation of signals on the surface from interactions within the bulk. 
	
	In terms of background sources, as mentioned electron interactions (electron recoils) are typically the result of background signals. Any incident neutron would be indistinguishable from a WIMP, so the detector is shielded using active (veto detectors) and passive (thick shield material, underground) methods.
	
	\section{Neutrinoless double beta decay}
	
		\subsection{a.)}
		
			Observation of a neutrinoless double beta decay would provide evidence that the neutrino is a majorana particle, in that it is it's own antiparticle. To observe such an interaction, the best process to study would be a well understood system that produced two neutrinos, i.e. $2\nu\beta\beta$. If the neutrino is a majorana, the neutrinoless process would be observed which would have total lepton number violation and no missing energy. 
			
			Observation of such a process, in addition to confirming the majorana nature of the neutrino, would allow for insight and observation on the absolute mass scale of neutrinos and also their hierarchy.
			
		\subsection{b.) }
		\newpage
		
		\subsection{c.) }
		
		The two principle approaches in the search are the 'Source $=$ Detector' and 'Source $\neq$ Detector' methods.
		
		'Source $=$ Detector'
		\begin{itemize}
			\item Pros: Can make use of high masses, highly efficient in detection of both produced electrons, reasonable resolution of energy.
			\item Cons: No way of extracting topology of event, and each detector is constrained to its source isotope.
		\end{itemize}
		
		'Source $\neq$ Detector'
		\begin{itemize}
			\item Pros: Can extract decay topology, which also aids in suppression of background events. Multi-purpose in terms of source isotope. 
			\item Cons: Smaller masses used, worse efficiency on electron detection and poor energy resolution.
		\end{itemize}
		
		\subsection{d.) }
		The CUORE, Cryogenic Underground Observatory for Rare Events, experiment is a 		'Source $=$ Detector' experiment, consisting of crystals of TeO, used as they have high concentrations of $^{130}$Te.  $^{130}$Te is used as it is the most abundant of any isotope that exhibits $0\nu\beta\beta$ decay, and possesses a significantly high enough $Q$-value to take the endpoint of the decay chain above the natural backgrounds. The crystals are kept down at the millikelvin level, so at this temperature small energy deposits within the crystal, such as those that could be the results of $0\nu\beta\beta$ produce measurable changes in the temperature of the crystal, which is then measured using a thermistor. the experiment is shielded by archaeological lead, along with moderators for neutrons and additional shielding provided by the cryostat. While the energy resolution is good, and the nodular crystal design allows for scalability, the experiment is hindered by contamination, especially near the surfaces of crystals, the difficulty of reading a signal and the complex nature of the cryostat.
		
	\section{Leptonic CP violation}
	
	The principle parameter of neutrino oscillation, (which in itself is the foundation from which to investigate CP violation) which allows investigation of CP violation is the mixing angle $\theta_{13}$. This parameter is primarily determined from the transition between $\nu_\mu\rightarrow\nu_e$.
	
	This is primarily studied using reactor neutrinos, simply by comparing behaviour of neutrinos and anti neutrinos. Backgrounds include intrinsic electron neutrinos, neutral current interactions and $\pi_0$ background events. For intrinsic electron neutrinos which can include solar neutrinos and neutrinos from muon decay in the focused beam, the background can be reduced by introducing veto detector methods, in which external interactions will trigger multiple detectors while target interactions will trigger only the target. The external detectors can also provide estimates as they are out of the beam, so provide background calculations. For the intrinsic muon contamination, as the muon decay is a three body interaction, it will have a broad spectrum of energies, so constraining observed interactions to a narrow e band helps reduce these effects. As a final remark, shielding methods can be used to reduce background readings, but this is somewhat problematic in some cases owing to the use on reactors as the neutrino source. For background events, neutrino oscillation experiments rely on long baseline experiments with detectors at origin and termination points. With two separate detectors the initial detector can be used to extrapolate a background rate for use in the target final detector. In addition, modern Multivariate Analysis methods using neural nets and BDTs can  help refine the phase space to desired events.
	
	\section{NC/CC Interactions}
	
	\newpage
	
	\section{Mass Hierarchy}
	
		As has been well covered, there are three distinct flavours of neutrino. While originally considered to be massless, oscillation has proved they possess mass. As the name suggests the hierarchy refers to the order of the masses, which state is heavier or lighter than any other, used as a related precursor to measurements of absolute mass.
		
		In terms of its determination, the flavour eigenstates are typically considered as superposition of the mass eigenstates, with the mixing matrix (PMNS) parametrized in terms of three mixing angles and a CP phase. In addition, calculations derived from the PMNS matrix lead to $\Delta m_{ij}^2$ terms for the mass squared splittings between each mass state.]
		
		In short, determination of these terms, which can be done using a variety of experimental techniques concerning neutrino oscillations (covered in prior questions), leads to insight into the mass hierarchy of the neutrinos, which is crucial to continually fleshing out the standard model and work to constrain or eliminate NP models.
	
	
	
	



	
	
	
	
	


\end{document}
