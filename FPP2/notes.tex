\documentclass[]{article}

\usepackage{amsmath}
\setlength{\parindent}{0em}
\setlength{\parskip}{1em}
\usepackage[none]{hyphenat}
\usepackage{fancyhdr}
\lhead{PNP 2012}
\pagestyle{fancy}

\usepackage[top=1.5cm, bottom=1.5cm, left=3cm, right=3cm]{geometry}
\begin{document}
\large

\section*{Particles}

\begin{itemize}
	\item $B^0: d\bar{b}, \bar{B^0}: b\bar{d}, B^+: u\bar{b}, B^-: b\bar{u}, B_s^0: s\bar{b}, \bar{B_s^0}: b\bar{s}$
	\item $C^0: c\bar{u}, \bar{D^0}: u\bar{c}, D^+: c\bar{d}, D^-: d\bar{c}, D_s^+: c\bar{s}, D_s^-: s\bar{c}$
	\item $\pi^+: u\bar{d}, \pi^0: u\bar{u} / d\bar{d}, \pi^-: d\bar{u}$
	\item $K^+: u\bar{s}, K^0: d\bar{s}, \bar{K^0}: s\bar{d}, K^-: s\bar{u}$
	
\end{itemize}

\section{Topics}

\subsection{CP Violation Discovery}

	C violation observed by pion decay and also by the noon existence of right handed neutrinos, see goldhaber. P violation observed by the Wu experiment. CP violation first proven by decays of neutral kaons in the cronin fitch experiment.

\section{Experiments}
 
 
 \subsection{AMS-1: Alpha Magnetic Spectrometer)} 
	 Designed to measure antimatter in cosmic rays, but not those from atmospheric showers, the original particles. Measured deflection, time of flight and energy loss to determine sign, velocity and charge. Looking for anti helium. Didn't find any. also absence of any high intesity radiation caused by interacting frontiers of antimatter pockets.
	 
	\subsection{GoldHaber}
	Produced neutrinos using electron capture in a $^{152}Eu$ source, chosen owing to spin 0 in both initial and final states. The experiment used a magnet to select photons emitted following the de excitation of the beta decay product of the EU source. This magnet was repeatedly switched between polaristations to select alternating sections of right and left handed photons. These right and left handed photons would require the neutrino produced in the decay to have same helicity, owing to conservation of quantum numbers in the interaction.
	
	The magnet focusing the photons focused them onto a scintillator composed of the EU decay product. It should be noted the magnet selection does not prevent differently handed photons, merely make uninihibited passage less likely, and as such excitation of the scintillator less likely. A positive hit on the scinitillator was indicative of a photon matching the polarity of the magnet, and therefore a corresponding  neutrino. The experiment measured a disparity between the helicity of the photons, with more left handed phtons detected, indicating more left handed neutrinos.
	
	\subsection{Liquid Argon Time Projection Chambers} 
	The basis of a time projection chamber is some substrate, which when a particle passes through, ionisation is caused along the particle track. The chamber is arranged as a drift chamber, with EM fields arranged such that charge particles 'drift' towards some form of (xy) position sensitive electron collector. (mesh of wired, EMTs). The time it takes for the electrons to drift to the detector is used to establish a z coordinated for the interaction.	
	Liquid Argon is the current favoured substrate for various reasons. being a noble gas, drift electrons are unlikely to be absorbed as they move to the detector, the high density of argon make interactions more likely in the substrate, and the fact that liquid argon scintillates in propotion to the energy deposited by the transient particle means additional information can be gleaned about the particle, and in addition the scintillation can be used as a marker for the drift, allowing a time delta to be calculate.
	
	\subsection{Liquid Xe Dark Matter}
	
	typically use a dual phase TPC, with gaseaous and liquid Xe as the substrate. Any wimp interaction would produce scintillation light, as Xenonn is good at that. The use of two phases is to induce a secondary scintillatio on transit of ionisation products between the two phases. This provides a second signal, and with known drift speeds from the TPC, a third dimension can be added to the interaction, with xy being deduced from the PMTs. These experiments are shielded in multiple fashions, the nature of the detector allows for self shielding, the leading example is at the Gran Sasso underground to sheild from cosmic rays, and the system is shielded with copper lead(archaeological oead is best) for neutron shielding. components are radiologically pure and clean.
	
	Whetherthe interaciton occured between a nuclei or a electron can be determined by the different scintillation energies 
	
	\subsection{CDMS}
	
	CDMS consists of disks of silicon and germanium, cooled to millikelvins. A
	voltage is applied across these discs. The discs are semiconductors so do not
	conduct electricity at low temperatures. The interaction of a WIMP would
	ionize the discs; this would be observed as a current pulse. The WIMP
	interaction would also cause a small temperature rise in the disk (a phonon)
	which can be detected with a device such as a thermistor.
	The main background sources are natural radioactivity, interactions of
	cosmic rays, and in particular interactions of neutrons (which can be produced
	by cosmic rays striking material surrounding the detector).
	12
	Natural radioactivity is eliminated by choosing the detector materials to
	be highly pure, and constructing and operating the detector in a high quality
	cleanroom. Much of the remaining radioactive background can be mitigated
	by comparing the sizes of the ionization and phonon signals; electron recoils
	give different signal ratios from nuclear recoils.
	Cosmic rays are mitigated by running the detector deep underground. In
	addition, a detector can be surrounded by active shielding (for example a
	scintillator or water detector) to spot the cosmic rays: this also flags times
	when a neutron is likely to have been produced. [Some detectors (such as the
	xenon experiments) also have the ability to self-shield: by reconstructing the
	location of the interaction, one can tell whether the interaction was near the
	edge of the detector (the likely location for neutron interactions) or towards
	the centre where less background is expected.
	
	\subsection{KamLAND}
	
	KamLAND consists of an 18 m diameter stainless steel spherical vessel with 1879 photomultiplier tubes mounted on the inner surface. Inside the sphere is a 13 m diameter nylon balloon filled with liquid scintillator. Outside of the balloon, non-scintillating, highly purified oil provides buoyancy for the balloon and acts as a shield against external radiation. Surrounding the stainless steel vessel is a water Cherenkov detector, which acts as a muon veto counter and provides shielding from radioactivity in the rock. very close to numerous reactors (baseline 180km). Relies on incident $\bar{\nu_e}$ of around 3MeV. The antineutrino undergoes an inverse beta decay reaction via the charge current interaction. This produces a neutron and a positron. The positron annihilates in a prompt event producing normally a pair of photons, while the neutron  after a short delayis captured by hydrogen, interact with a proton to produce a delayed photon

	\subsection{Cronin-Fitch}
	
	Looking for CP violating kaon to 2 pion decays decays from neutral kaon production. The kaons for the experiment were produced from pion beams, forming the standard CP eigenstates for the Kaons. These eigenstates had predicted decay interactions (1->2, 2->3) to pions presuming CP convse ation in teh weak interaction. The desired effect was to try to find a $K_2$ decaying to two pions rather than 3. There were two interesting kinematic facets aiding this. Firstly, to achieve a beam of solely $K_2$, the detector was placed a short distance away from the production point. $K_1$ has a much shorter lifetime, so by introducing this extra distance and therefore time, an almost pure beam could be achieved with all the $K_1$ having decayed. When the $K_2$ decayed into three pions, there would be some angular difference, and they would not be along the beam line. The Cronin-fitch experiment suggested the existence of the CP violating two pion decay firstly by observing an excess of two pion events relative to those expected given some small amount of $K_1$ survival, and also by an excess of events occuring along the beam line. The effect was minuscule, but evidence of small small violation of CP. 
	
	\subsection{MINOS}
	
	Near Far detector, as typical for neutrino experiment. Proton pulses against a graphite target to produce pions and kaons, focused using magnertic horns into a beam. Subsequent decays form a beam of primarily muon neutrinos towards eh far detector, undergoing oscillation along the way. Mineestoa based, 700km baseline 700m below surface. 
	
	The MINOS detectors consist of interleaved steel and scintillator planes. A
	neutrino interacts in the detector (most likely in the steel). This produces
	charged particles, which pass through scintillator causing an emission of light.
	This light is transported, via optical fibres, to photomultiplier tubes. The
	scintillator planes are divided into strips so we can tell where the charged particles
	passed through; scintillator strips on successive planes are oriented at
	90◦
	to each other, allowing three-dimensional reconstruction of the neutrino
	interaction.
	
	There are two primary reasons why MINOS and T2K struggle to see tau
	neutrino appearance. The first is that the detectors are not optimised to
	identify tau leptons (their granularity is too coarse). The second is that
	the 3.5 GeV threshold for the ντ interaction means that most of the νµ →
	ντ transition happens below the threshold energy, so the majority of the
	resultant tau neutrinos cannot interact in the first place.
	
E ∼ 3 GeV

MINOS, T2K, NOνA Sensitive to νµ and νe. Can measure ∆m2
32, sin2
(2θ23),
sin2
(2θ13)


eVENT TOPologies: mu long muon track with hadronic certex activity. electron compact showers nc event more diffuse showers.

\subsection{LHCb}

Search for new physics by probing flavour structure of SM. Study CP violation and rare decays with beauty and charm hadrons.  

Detector features: excellent tracking and particle identification, principle feature is the limited rapidity. The detector is designed to spot events within the forward region of the detector. Designed to deal with b and c hadrons, these are most often produced in the forward region of the detector, so concentrating the detection around that area allows for more signal efficiency. 

There is strong vertex detection. With studying B mesons, there is generally both primary and secondary verticies in teh interactions, so being able to preciles map those is ideal. In addition there is high quality particle id. ost of teh stated analysis goals for the experiment rely on spotting decay producs. 

\subsection{Neutrinoless Doublebeta decay}

Forbidden in SM as the process violates total lepton number conservation. Observation of the decay would suggest that neutrinos are there own anti particle, in that being majorana particles. in addition, to self annihilate it must flip helicity and this would mean the decay rate would be dependent on the absolute neutrino mass.

As proven by oscillations, the neutrino flavours have mass, and thehierarchy refers to the order of those masses with relation to each other. 

\subsection{T2K}

Uses superkamiokande cherenkov detector, nar far detector setup, muon neutrino beam, measures oscillations driven by $\Delta m^2_{32}$ measuring the angles 23 and also 13. background from neutral pion double photon overlap


\subsection{wu experiment}

Designed to test party conservation in beta decay. Use cobalt 60, monitor in a uniform magnetic field. Cobalt decays using proton beta decay but forms an excited state that then emits 2photons. The tric was controlling the polarisation fo the cobalt. This required high fields at low temperatures, hence the use of liquid helium, (adiabatic demagnitization, magnetic cooling).

The analysis came from comparing thre distribution of phtoon emissions with the electron emissions. if they were comparable, there was no bias in the direction of the electrons. Adjusting polarisation levels and direciton of magnetic field had significant effects on the counting rate, with all the dat suggesting left handed electrons only. This was in turn indicative of solely right handed anti neutrinos. Hence, P violation was true for neutrinos 

\subsection{Gallex Experiment}

Gallium acid sollutiuon, interact wiht electron neutrinos to produce an electron and Ge, saw half of exected solar flux.

\subsection{Homestake}

Chlorin liquid, looking to capture solar neutriono elecrton. which interact to produce argon, argon was fildered out and decays were looked for. saw one third predicted flux.

\subsection{SNO}
Lots of heavy water cherenkov detector neutrinos interact with deutreron inCC to produce cherenkov liht from electrons and protons. 

Elsatric scattering off electrons with CC and NC produce cernenkov and dominated by electron neutrino. Sensitive to neutriono ditection.

Later salt was added, which has a high neutron absorbtion cross section and as such when neutral currenent interaction occured released phtons. 

Sno found total number of neutrinos was conserved NC sensitive to all flavours, while they were cahngin flavour/

\subsection{DAYA BAY}
measure theat 13

6 detectors, scintillator configured for high neutron capture. Recrods inverse beta decay of antti electron neutrinosproducing a propt event froom positron annihilation and a delated event from enturon capture.measured 13 

\subsection{KATRIN}
 
 beta decay endpoint  for neutrino mass. retarding electric field. Trituim source, beta decay e guided to spectrometer counted proptional to intergrate beta spectrum. Use tritium for exact nuclear modelling potential, account for recoil energy of the nucleus and work at cryogenic temepratures to limite thermal motion.
 
 Tritium: low amount of retardation, KTatrin massive to reach 1V resolutionsensitive to 0.2eV neutrino masses. Backghrouns is tritiium and other imputities in the spectromet and cosmic rays
 
 \subsection{CUORE}
 		The CUORE, Cryogenic Underground Observatory for Rare Events, experiment is a 		'Source $=$ Detector' experiment, consisting of crystals of TeO, used as they have high concentrations of $^{130}$Te.  $^{130}$Te is used as it is the most abundant of any isotope that exhibits $0\nu\beta\beta$ decay, and possesses a significantly high enough $Q$-value to take the endpoint of the decay chain above the natural backgrounds. The crystals are kept down at the millikelvin level, so at this temperature small energy deposits within the crystal, such as those that could be the results of $0\nu\beta\beta$ produce measurable changes in the temperature of the crystal, which is then measured using a thermistor. the experiment is shielded by archaeological lead, along with moderators for neutrons and additional shielding provided by the cryostat. While the energy resolution is good, and the nodular crystal design allows for scalability, the experiment is hindered by contamination, especially near the surfaces of crystals, the difficulty of reading a signal and the complex nature of the cryostat.
 		
 		\subsection{CDMS} 
 		
 		millikelvin disks of germanium and silicon, measure inozation and phonons froma WIMP, these can be distinguishedfgWfefeffffffFffwfffffffffefffefff
 		

\end{document}
