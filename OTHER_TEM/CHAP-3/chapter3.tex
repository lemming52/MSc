\chapter{Useful commands for drafting}\label{c:drafting}

\section{Introduction}

This Chapter discusses various commands -- some built into the memoir
class, others defined in \verb|thesis.sty|, and some requiring
additional packages to be loaded -- which may be useful when drafting.
These include:
%
\begin{enumerate}
%
\item marginal notes (see Section~\ref{s:notes});
%
\item marginal marks to indicate changes/deletions/additions (see
Section~\ref{s:changes});
%
\item the addition of line numbers (see Section~\ref{s:lineno});
%
\item adding a `watermark' on each page, to timestamp a particular
version (see Section~\ref{s:watermark}).
%
\end{enumerate}

\section{Marginal notes}\label{s:notes}

The \verb|thesis.sty| file defines the \verb|\note{...}| command for
marginal notes\note{this is an example marginal note} (in a small font,
black). In addition \verb|\todo{...}| and \verb|\tocheck{...}| are
defined for specific notes/reminders on items still to do or check,
which are shown as marginal notes in red and blue respectively, which
are illustrated below.

Here is some sample text.
%
\todo{here is a note of something still to do}
%
Here is some sample text. Here is some sample text. Here is some sample
text. Here is some sample text. Here is some sample text. Here is some
sample text. Here is some sample text. Here is some sample text. Here is
some sample text.
%
\tocheck{here is a note about something to check}
%
Here is some sample text. Here is some sample text. Here is some sample
text. Here is some sample text. Here is some sample text. Here is some
sample text. Here is some sample text. Here is some sample text. Here is
some sample text. Here is some sample text.

\section{Changes/deletions/additions}\label{s:changes}

The memoir class provides commands to mark changes, deletions or
additions made, i.e.\ \verb|\changed{...}|, \verb|\deleted{...}| and
\verb|\added{...}| (see Chapter~18 of the 
\href{http://www.ctan.org/pkg/memoir}{memoir} user manual for
details). These add symbols with a text label to the margin to indicate
revisions have been made (that is provided \verb|\changemarks| rather
than \verb|\nochangemarks| is used, and that the \verb|draft| option is
used, or \verb|\draftdoctrue| is set -- otherwise these change marks are
not shown).

Here is some sample text.\deleted{label} Here is some sample text. Here
is some sample text. Here is some sample text. Here is some sample text.
Here is some sample text. Here is some sample text.
Here is some sample text.\added{text} Here is some sample text.
Here is some sample text. Here is some sample
text. Here is some sample text. Here is some sample text. Here is some
sample text.\changed{label text} Here is some sample text. Here is some
sample text.

In addition, \verb|thesis.sty| defines a command \verb|\fixed{...}|
which can be used to mark changed text, in green (along with `***' 
added in the
margin).

Here is some sample text. Here is some sample
text. \fixed{Here is some `fixed' text. Here is some `fixed' text. Here
is some `fixed' text.} Here is some sample text.

\section{Additional packages for drafting}

\subsection{Add line numbers}\label{s:lineno}

If you want to number the lines when drafting, then you can use the
`lineno' package. This is included towards the end of \verb|thesis.sty|,
but is commented out. If you want to use this package, the uncomment it
in \verb|thesis.sty|. Then you can switch linenumbers on or off with
\verb|\linenumbers| or \verb|\nolinenumbers|. You can reset the line
numbers -- for example if you want to number the lines in each chapter
from one -- using \verb|\resetlinenumbers|.

\subsection{Add a `watermark'}\label{s:watermark}

If you want to add a `watermark' to each page, to indicate the time/date
of the draft version, you can use the `draftwatermark' package. This is
included towards the end of \verb|thesis.sty| -- along with commands to
setup a faint grey `watermark' in the left margin of each page saying
`draft' with the date and time -- but is commented out. Uncomment and
edit as needed the relevant lines in \verb|thesis.sty| if you want to
use a watermark when drafting.

\endinput
